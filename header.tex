\usepackage[a4paper,margin=2cm]{geometry}
\usepackage{enumitem}
\usepackage[rflt]{floatflt}
\setlist{nosep}

\usepackage{standalone}

% Automata stuff
\usepackage{tikz}
\usetikzlibrary{automata}
\usetikzlibrary{arrows}
\usetikzlibrary{quotes}
\usetikzlibrary{calc}

\tikzstyle{automaton}=[
  % Use nice arrows that do not touch their destination.
  semithick,shorten >=1pt,>=stealth',
  % Encourage a common distance between all states.
  node distance=2cm,
  % Disable the "start" text in front of the initial arrow.
  initial text=,
  % Reduce the size of the hidden node at the beginning of the initial arrow.
  every initial by arrow/.style={every node/.style={inner sep=0pt}},
  % Encourage a common size of all states that is smaller than the default.
  every state/.style={minimum size=7.5mm,fill=white}
]

\tikzstyle{state-labels}=[state/.style=state with output,inner sep=2pt]

% State names are labels displayed below a state
\tikzstyle{statename}=[
  below,label distance=2pt,
  fill=yellow!30!white,
  rounded corners=1mm,inner sep=2pt
]

% Acceptance sets are displayed as small dark blue circle with white
% sans-serif number.  Use anchor=center to ignore the predefined
% anchor on edges like "loop above" or "loop left".
\tikzstyle{accset}=[
  fill=blue!50!black,draw=white,text=white,thin,
  circle,inner sep=.9pt,anchor=center,font=\bfseries\sffamily\tiny
]

\tikzstyle{loop 30}=[in=15,out=45,loop,right]
\tikzstyle{loop 45}=[in=30,out=60,loop,right]
\tikzstyle{loop 135}=[in=120,out=150,loop,left]
\tikzstyle{loop -30}=[in=-45,out=-15,loop,right]
\tikzstyle{loop -45}=[in=-60,out=-30,loop,right]
\tikzstyle{loop -60}=[in=-75,out=-45,loop,right]
\tikzstyle{loop -120}=[in=-135,out=-105,loop,left]
\tikzstyle{loop -135}=[in=-150,out=-120,loop,left]


% Automaton environment with
% 150% scaling, because SVG output is too small otherwise.
\usepackage{environ}
\NewEnviron{automaton}[1][]%
{\hfill\smash{{\begin{tikzpicture}[automaton,baseline=(current bounding box.north),#1]
\BODY
\end{tikzpicture}}}\vspace*{-2em}}

\let\G\undefined
\let\U\undefined
\let\M\undefined
\DeclareMathOperator{\F}{\textup{\textsf{F}}}
\DeclareMathOperator{\G}{\textup{\textsf{G}}}
\DeclareMathOperator{\X}{\textup{\textsf{X}}}
\newcommand{\U}{\mathbin{\textsf{U}}}
\newcommand{\R}{\mathbin{\textsf{R}}}
\newcommand{\W}{\mathbin{\textsf{W}}}
\newcommand{\M}{\mathbin{\textsf{M}}}
